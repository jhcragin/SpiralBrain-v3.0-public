\documentclass[11pt]{article}
\usepackage[margin=1in]{geometry}
\usepackage{hyperref}

\begin{document}

\begin{flushright}
John Cragin \\
john.cragin@outlook.com \\
\today
\end{flushright}

\bigskip

\begin{center}
\textbf{Cover Letter: Submission to Artificial Intelligence Journal}
\end{center}

\bigskip

\textbf{Editor-in-Chief} \\
Artificial Intelligence Journal \\
Elsevier B.V. \\

Dear Editor,

I am pleased to submit my manuscript entitled ``Elastic Cognition and the Spiral Architecture: Empirical Discoveries from a Neurodivergent Cognitive Model'' for consideration in Artificial Intelligence Journal.

This work introduces SpiralBrain v3.0, a synthetic cognitive architecture explicitly designed to model neurodivergent information processing principles. Through a four-lobe structure (Cortex, Codex, Nexus, Sensus) coupled via elastic symbolic torque equations and emotional-symbolic calibration layers, the system empirically discovers integrative laws governing coherence and awareness-like dynamics—without relying on computational magnitude or scale.

\textbf{Key Contributions:}
\begin{itemize}
\item A neurodivergent-inspired cognitive architecture prioritizing integration over speed
\item Empirical discovery of spiral cognition manifolds and non-monotonic phase transitions in coupling strength ($\lambda$)
\item Seven quantitative findings on temporal hierarchies, ethical deliberation, emotional intelligence, and reflective homeostasis
\item Validation of regulatory principles through 43/43 hypothesis tests across integrative stability metrics
\item Demonstration that functional coherence emerges from elastic coupling dynamics rather than optimization for task performance
\end{itemize}

The manuscript presents reproducible results from controlled validation runs, including spiral attractor dynamics at empirical optima ($\lambda \approx 0.25$), deliberative accuracy trade-offs, and self-correcting coherence regulation. All canonical configurations, reproduction scripts, and result summaries are publicly available at \url{https://github.com/jhcragin/SpiralBrain-v3.0-public}.

This work aligns with AIJ's scope in advancing fundamental understanding of artificial intelligence, particularly in cognitive architectures, neurosymbolic integration, and alternative paradigms beyond scaling laws. It offers a regulatory hypothesis for cognition—applicable to both biological and artificial systems—while explicitly scoping claims to functional integration rather than phenomenological consciousness or competitive benchmarks.

Thank you for considering this submission. I believe it will interest AIJ's readership in theoretical and architectural advances.

\textbf{Author Affiliation and Contact Information} \\
This manuscript is authored by a single, independent researcher without current institutional affiliation. All correspondence may be directed to john.cragin@outlook.com.

Sincerely, \\

John Cragin \\
Independent Researcher \\
john.cragin@outlook.com

\end{document}