\documentclass[11pt,a4paper]{article}

\usepackage[strings]{underscore}
\usepackage[utf8]{inputenc}
\usepackage[T1]{fontenc}
\usepackage{amsmath,amssymb,amsthm}
\usepackage{graphicx}
\usepackage{float}
\usepackage{hyperref}
\hypersetup{
  colorlinks=true,
  linkcolor=blue,
  citecolor=blue,
  urlcolor=blue
}
\usepackage{natbib}
\usepackage{tabularx}
\usepackage{geometry}
\usepackage{setspace}
\usepackage{booktabs}
\usepackage{subcaption}
\usepackage{xcolor}
\usepackage{tikz}
\usepackage[strings]{underscore} % Added from v4 for compatibility

\geometry{margin=1in}

\title{Elastic Cognition and the Spiral Architecture: Regulation-First Synthetic Cognition Under Uncertainty}

\author{John Cragin \\
Independent Researcher \\
\texttt{john.cragin@outlook.com}}

\date{\today}

% Symbol Macros
\newcommand{\coh}{\Phi}            % coherence
\newcommand{\cohprime}{\Phi'}      % derivative of coherence
\newcommand{\coupling}{\lambda}    % coupling constant
\newcommand{\phaselock}{\phi_{\text{lock}}} % phase-lock angle
\newcommand{\secd}{\text{SEC}_{\text{drift}}} % SEC drift

\begin{document}

\maketitle

\begin{abstract}
SpiralBrain-v3.0 is a synthetic cognitive organism designed to function as a measurement instrument for internal cognitive regulation. Unlike performance-optimized artificial intelligence systems, SpiralBrain-v3.0 is not evaluated primarily by task accuracy or benchmark scores. Instead, it is constructed to enable direct observation of how coherence, stability, ethical constraint, and recovery are maintained within a continuously operating cognitive system under uncertainty.

The architecture employs elastic coupling among semi-independent functional subsystems and a regulatory mechanism that converts affective state into measurable control signals governing timing, gain, and constraint. Task environments and standard benchmarks are used as structured perturbations to stress the system, allowing internal dynamics to be empirically measured rather than inferred from output behavior alone.

Across a series of controlled experiments, SpiralBrain-v3.0 exhibits stable internal attractors, non-trivial recovery dynamics, and emergent temporal hierarchies governing regulation and constraint. Notably, optimal coupling regimes and ethical stabilization arise from system dynamics rather than from predefined rules, learned policies, or external supervision. These behaviors are evaluated using internal state metrics, including coherence, drift, phase alignment, and recovery time.

Neurodivergent cognition is referenced strictly as an architectural abstraction, reflecting nonlinear, parallel, and elastically coupled cognitive dynamics rather than any clinical or population-level model. No claims are made regarding consciousness, human cognition, or diagnostic inference.

Taken together, these results demonstrate the feasibility of regulation-first synthetic cognition and position SpiralBrain-v3.0 as an experimental platform for studying internal cognitive physiology, offering an alternative framework for resilient and ethically constrained artificial intelligence design.

\textbf{Keywords:} elastic cognition, neurodivergent modeling, spiral architecture, integrative dynamics, empirical cognitive discovery, synthetic coherence, cognitive architecture, regulatory intelligence, non-learning AI, cognitive physiology, uncertainty tolerance
\end{abstract}

\section{Introduction}
\label{sec:introduction}

\subsection{Cognitive Instability Under Uncertainty} % Added from v2/v3
Artificial intelligence systems optimized for performance metrics often exhibit fundamental instabilities when confronted with uncertainty. These instabilities manifest as internal failure modes that remain invisible under standard evaluation protocols:

- \textbf{Confidence collapse}: Systems that perform well on training distributions lose calibration when faced with out-of-distribution inputs or ambiguous conditions
- \textbf{Silent internal drift}: Parameter evolution and representation shift occur without external observability, leading to unpredictable behavior
- \textbf{Hidden cognitive pathology}: Correct outputs mask deteriorating internal state, creating systems that appear functional while suffering from fundamental architectural flaws

These failure modes are particularly dangerous in high-stakes applications where system confidence must remain coupled with actual capability. Traditional performance-centric evaluation frameworks—focused on accuracy, precision, and recall—fail to detect these internal instabilities because they measure outputs rather than cognitive health.

\subsection{Limitations of Performance-Centric Evaluation} % Added from v2/v3
The dominant paradigm in AI evaluation treats accuracy as a proxy for system quality. This approach contains several critical limitations:

\textbf{Accuracy does not imply safety}: A system can achieve high accuracy while suffering from internal instabilities that make it unreliable under stress. The coupling between performance and stability is assumed rather than verified.

\textbf{Accuracy does not imply intelligence}: Task-specific performance says little about the cognitive mechanisms underlying that performance. Statistical correlation with training data can masquerade as reasoning capability.

\textbf{Accuracy does not imply cognitive health}: Internal regulatory processes—homeostasis, coherence maintenance, and hazard detection—may degrade even as external performance remains acceptable. This creates a dangerous disconnect between observable behavior and underlying cognitive viability.

\subsection{What SpiralBrain-v3.0 Is — and Is Not}
SpiralBrain-v3.0 is a synthetic cognitive organism designed to function as a measurement instrument for internal cognitive regulation. It is an actively executing system with persistent internal state, regulatory dynamics, and recovery behavior, constructed to allow direct observation of how coherence, stability, and constraint are maintained under uncertainty.

SpiralBrain-v3.0 is not a product system, a benchmark-optimized model, or a human cognitive simulation. It is not designed to maximize task accuracy, emulate biological cognition, or compete with large-scale machine learning systems. Instead, it serves as an experimental platform for studying regulated cognitive dynamics—how a cognitive system preserves internal integrity when exposed to stress, scarcity, ambiguity, or conflicting demands.

In SpiralBrain-v3.0, task environments and benchmarks are used as structured perturbations, not optimization objectives. Performance outcomes are observed, but they are not the primary target of the system's design. The primary object of study is the system's internal regulatory physiology—coherence maintenance, drift containment, and recovery mechanisms.

\subsection{Design Objective} % Added from v2/v3
SpiralBrain v3.0 addresses these limitations through a fundamentally different design philosophy. Rather than optimizing for task performance, the architecture prioritizes cognitive integrity as a prerequisite for sustained operation. This manifests in three core design objectives:

- \textbf{Regulatory primacy}: Cognition is framed as a regulated dynamical process rather than an optimization target
- \textbf{Uncertainty tolerance}: Systems must endure incorrect outcomes without internal destabilization
- \textbf{Observability without modification}: Internal cognitive state can be monitored and logged without altering the cognitive organism itself

This approach treats intelligence not as the ability to produce correct answers, but as the ability to maintain cognitive coherence under uncertainty. The result is an architecture that can be wrong in interesting ways—revealing the boundaries of cognitive viability rather than masking them.

\subsection{Research Question and Design Motivation}
The central research question is: How can a cognitive system preserve internal coherence, ethical constraint, and recoverability under uncertainty without relying on learning, scale, or external supervision?

SpiralBrain prioritizes cognitive integrity through regulatory primacy, uncertainty tolerance, and observability. Neurodivergent cognition is referenced as an architectural abstraction, reflecting nonlinear, parallel, and elastically coupled dynamics.

\subsection{Paper Overview} % Enhanced with v4
The following sections describe the system architecture, experimental methodology, quantitative findings, and theoretical implications.

\section{Related Work} % Added from v2/v3
\subsection{Learning-Based AI Systems}
Contemporary AI research predominantly focuses on learning-based systems that accumulate task-specific capabilities through experience. Supervised learning, reinforcement learning, and large language models all share a fundamental dependence on performance-centric evaluation metrics.

These systems inherently couple learning mechanisms with optimization objectives. Training protocols simultaneously modify internal parameters and evaluate external performance, creating an inseparable link between the learning process and the metrics used to assess it. This coupling makes it difficult to distinguish genuine cognitive advancement from statistical correlation.

The performance-centric evaluation paradigm that dominates these systems assumes that optimization for task accuracy will naturally produce systems with robust cognitive capabilities. However, this assumption breaks down under uncertainty, where the correlation between training performance and real-world stability becomes tenuous.

\subsection{Cognitive Architectures}
Cognitive architectures have traditionally emphasized symbolic processing and hybrid approaches combining symbolic reasoning with subsymbolic computation. Systems like SOAR, ACT-R, and CLARION focus on explicit reasoning processes and task execution capabilities.

While these architectures provide more interpretable reasoning chains than neural network approaches, they share limitations in internal state instrumentation. Most cognitive architectures treat internal cognitive processes as implementation details rather than primary evaluation targets. This limits their ability to detect subtle forms of cognitive drift or regulatory failure.

Recent work in neuro-symbolic systems attempts to bridge this gap by combining the interpretability of symbolic systems with the pattern recognition capabilities of neural networks. However, these approaches still primarily evaluate based on task performance rather than internal cognitive health.

\subsection{Cybernetics and Regulation}
The field of cybernetics provides foundational concepts for understanding cognition as a regulated process. Viability theory and homeostasis research demonstrate that sustained operation requires active regulatory mechanisms rather than passive optimization.

Ashby's law of requisite variety suggests that regulatory systems must have sufficient complexity to handle environmental uncertainty. This principle applies directly to cognitive systems, where the ability to maintain internal stability under external perturbation becomes a core capability.

Recent work in cognitive cybernetics explores how biological nervous systems maintain stability through regulatory feedback loops. These insights inform the design of synthetic cognitive systems that prioritize viability over performance. However, most cybernetic approaches to AI still focus on control systems rather than comprehensive cognitive architectures.

\section{Worked Example: Observing Regulation Under Perturbation} % Added from v2/v3
\label{sec:worked_example}

To illustrate SpiralBrain-v3.0's regulation-first design, we present a concrete example of the system responding to cognitive stress. This example demonstrates how internal metrics reveal regulatory dynamics that external performance measures alone would miss.

\subsection{Experimental Setup}
We subjected SpiralBrain-v3.0 to a modified MMLU benchmark task involving ethical reasoning under time pressure. The system was initialized with \(\coupling = 0.25\) (the empirically discovered attractor) and exposed to a sequence of 50 questions requiring moral judgment. To induce stress, we artificially accelerated the temporal hierarchy by reducing \(\tau_{\text{ethics}}\) from 10 to 3 relative to \(\tau_{\text{emotion}}\), forcing faster constraint enforcement.

\subsection{Internal Dynamics Observed}
Figure \ref{fig:regulation_example} shows the system's response over 200 processing cycles. Initially, coherence \(\coh\) remains stable at 0.85, with SEC drift below 0.10, indicating regulated operation. At cycle 50, the temporal acceleration triggers instability: \(\coh\) drops to 0.62, SEC drift spikes to 0.28, and phase alignment degrades.

\begin{figure}[htbp]
\centering
\includegraphics[width=\textwidth]{../figures/neurodivergent_validation.pdf}
\caption{Internal regulation dynamics during ethical reasoning stress test. Coherence (\(\coh\)) shows initial stability, perturbation-induced decline, and recovery. SEC drift indicates emotional regulatory response. Task accuracy (dashed line) correlates with but does not drive internal stabilization.}
\label{fig:regulation_example}
\end{figure}

Crucially, the system initiates recovery without external intervention. By cycle 120, \(\coh\) recovers to 0.78, SEC drift normalizes to 0.12, and phase-lock improves. This recovery occurs through endogenous adjustment of coupling strength, increasing from 0.25 to 0.31.

\subsection{Code Snippet: Measuring Recovery}
The following Python code demonstrates actual recovery measurement from SpiralBrain's metacortex event stream:

\begin{verbatim}
import json
import numpy as np
from pathlib import Path

def measure_recovery_from_events(jsonl_path: str, window_size: int = 50):
    """
    Measure cognitive recovery from SpiralBrain metacortex events.
    
    Args:
        jsonl_path: Path to metacortex_events.jsonl file
        window_size: Number of events to use for baseline calculation
        
    Returns:
        dict: Recovery metrics including time, stability, and hazard levels
    """
    # Load events
    events = []
    with open(jsonl_path, 'r') as f:
        for line in f:
            events.append(json.loads(line.strip()))
    
    # Extract time series data
    timestamps = [e['timestamp'] for e in events]
    coherence_values = []
    sec_drift_values = []
    hazard_scores = []
    
    for event in events:
        # Use state_magnitude as coherence proxy
        coherence_values.append(event.get('state_magnitude', 0.5))
        sec_drift_values.append(event.get('sec_drift', 0.0))
        hazard_scores.append(event.get('hazard_score') or 0.0)
    
    # Convert to numpy arrays
    coherence_history = np.array(coherence_values)
    sec_drift_history = np.array(sec_drift_values)
    hazard_history = np.array(hazard_scores)
    
    # Find perturbation onset (significant drop in coherence)
    baseline_coh = np.mean(coherence_history[:window_size])
    baseline_std = np.std(coherence_history[:window_size])
    perturbation_threshold = baseline_coh - 2 * baseline_std
    
    perturbation_indices = np.where(coherence_history < perturbation_threshold)[0]
    if len(perturbation_indices) == 0:
        return {"error": "No perturbation detected"}
    
    perturbation_start = perturbation_indices[0]
    
    # Track recovery (return to 90% of baseline)
    recovery_threshold = baseline_coh * 0.9
    post_perturbation = coherence_history[perturbation_start:]
    recovery_indices = np.where(post_perturbation >= recovery_threshold)[0]
    
    if len(recovery_indices) == 0:
        recovery_cycle = len(post_perturbation)  # No recovery
    else:
        recovery_cycle = recovery_indices[0]
    
    # SEC stabilization (low drift post-recovery)
    recovery_window = sec_drift_history[perturbation_start + recovery_cycle:
                                       perturbation_start + recovery_cycle + window_size]
    sec_stable = np.mean(recovery_window) < 0.15 if len(recovery_window) > 0 else False
    
    # Hazard assessment
    hazard_level = np.mean(hazard_history[perturbation_start + recovery_cycle:
                                         perturbation_start + recovery_cycle + window_size])
    
    return {
        "recovery_time_cycles": recovery_cycle,
        "emotional_stability": sec_stable,
        "hazard_level": hazard_level,
        "baseline_coherence": baseline_coh,
        "perturbation_start": perturbation_start
    }

# Example usage
results_file = Path("results/metacortex_events.jsonl")
recovery_metrics = measure_recovery_from_events(results_file)
print(f"Recovery time: {recovery_metrics['recovery_time_cycles']} cycles")
print(f"Emotional stability: {recovery_metrics['emotional_stability']}")
print(f"Hazard level: {recovery_metrics['hazard_level']:.3f}")
\end{verbatim}

\subsection{Narrative Interpretation}
This example demonstrates regulation-first cognition in action. External task accuracy (78\% correct) serves as a perturbation source, not an optimization target. The system's value lies in its observable internal regulation: detecting instability, modulating coupling, and achieving recovery. This behavior validates SpiralBrain-v3.0 as a measurement instrument for studying cognitive physiology under stress.

\section{System Architecture} % From v1, enhanced with v2/v3 non-learning and v4 details
\label{sec:architecture}

\subsection{Architectural Overview} % Added from v2/v3
\subsubsection{Core Design Principles} % From v2/v3
SpiralBrain v3.0 implements a synthetic cognitive architecture that explicitly rejects learning-based approaches. The system operates under strict constraints that prevent the emergence of traditional AI capabilities:

\textbf{No learning mechanisms}: The architecture contains no parameter optimization, gradient descent, or data-driven training processes. All behavior emerges from fixed architectural interactions rather than learned representations.

\textbf{No persistent or cross-run memory}: Each instantiation begins from an identical clean state. No information persists between runs, and no cross-run adaptation occurs within the cognitive organism.

\textbf{No cross-run state carryover}: The experimental process may learn and adapt, but the cognitive organism itself remains static across instantiations. This creates a clean separation between methodological improvement and cognitive evolution.

\textbf{Within-run regulatory adaptation only}: The system can adjust regulatory parameters during a single execution in response to immediate conditions, but these adjustments do not persist or influence future instantiations.

\textbf{Full observability without self-modification}: All internal cognitive variables can be monitored and logged in real-time without altering the cognitive processes themselves. This enables scientific study without observational artifacts.

\subsubsection{Multi-Pathway Cognitive Structure} % From v2/v3
The architecture implements cognition through constrained interactions between specialized processing pathways rather than unified problem-solving mechanisms. Four primary cognitive lobes—Sensus, Cortex, Codex, and Nexus—process different aspects of cognitive tasks while maintaining continuous cross-pathway modulation.

This multi-pathway approach creates behavior through the regulated interaction of specialized components rather than through independent problem-solving. The resulting cognitive dynamics emerge from architectural constraints rather than learned associations, enabling predictable behavior patterns without statistical training.

\subsubsection{Homeostatic Regulation} % From v2/v3
Regulatory mechanisms form the core of SpiralBrain's cognitive architecture. A tri-band elastic regulation system maintains cognitive stability through multiple feedback loops:

\textbf{Drift containment}: Active monitoring prevents cognitive variables from exceeding viable bounds, maintaining system coherence under perturbation.

\textbf{Cognitive hazard detection}: Internal monitoring identifies conditions that could lead to cognitive instability, enabling preemptive regulatory responses.

\textbf{Recovery dynamics}: Following perturbation, the system implements elastic recovery mechanisms that restore cognitive equilibrium without external intervention.

These regulatory processes operate continuously, treating cognition as a dynamical system that must maintain viability rather than optimize performance. The result is an architecture that can endure uncertainty without collapse.

\subsection{Operational Definitions and Validation Protocols} % From v4
To ensure reproducibility and clarity, we define key metrics and protocols as follows:

\textbf{Coherence \(\coh\)}. 
Let \( y_i[t] \) denote the scalar output (or fixed projection) of lobe \( i \) at timestep \( t \).
For each timestep \( t \), compute the Pearson correlation matrix \( C[t] \) over a trailing window of length \( W \).
Coherence is defined as

\[
\coh[t] = \frac{2}{N(N-1)} \sum_{i<j} \lvert C_{ij}[t] \rvert
\]

and the overall coherence is

\[
\bar{\coh} = \frac{1}{T} \sum_{t=1}^{T} \coh[t]
\]

normalized to the range \([0,1]\) by construction.

\textbf{SEC Drift:} The standard deviation of the Symbolic-Emotional Calibration vector components (valence, arousal, hazard, resonance, harmony, amplitude, coherence, and complexity) over a 100-step window. Thresholds were selected empirically as stability indicators and should be interpreted as operational definitions rather than universal constants. Window length (W) is treated as an operational parameter and reported alongside results.

\textbf{\(\coupling\)-Sweep Protocol:} Coupling parameter \(\coupling\) was swept across a bounded range using a fixed step size; perturbations were applied periodically as bounded noise injections. Exact sweep bounds and perturbation schedules are reported per experiment. Where applicable, comparative analyses were evaluated using standard statistical tests; detailed assumptions and test selection are reported alongside results.

\textbf{Stability/Failure Criteria:} System failure defined as \(\coh < 0.3\) sustained for \(>20\%\) of trial duration or recovery requiring external intervention. Awareness-like dynamics measured as \(\cohprime\) maintained above empirically selected thresholds for the majority of stable periods. We use “awareness-like” purely as a dynamical descriptor of sustained integrative responsiveness, not as a claim about subjective experience.

\subsection{Overview}
SpiralBrain v3.0 implements a \textit{four-lobe topology} that parallels functional divisions observed in human neurocognition.
Each lobe operates as a semi-autonomous processor coupled through elastic feedback channels governed by the SpiralCode recursive torque equation:

\[
\frac{d\coh}{dt} = \coupling (\tau^{-1} I - \coh) + \eta E
\]

where \(\coh\) is the instantaneous coherence state, \(\coupling\) is the coupling constant, \(\tau\) represents lobe-specific temporal hierarchy, \(I\) denotes incoming symbolic input, \(E\) is the emotional modulation vector, and \(\eta\) is an adaptive gain derived from SEC (Symbolic-Emotional Calibration) feedback.

The architecture was not designed to simulate a neural substrate, but to embody \textbf{cognitive principles}—integration, differentiation, reflection, and elasticity—within a symbolic-computational framework.

\subsection{Cognitive Lobes} % Enhanced with v4 details
\textbf{Cortex — Metacognitive and Ethical Regulation}  
Implements reflective homeostasis by monitoring global coherence \(\coh(t)\) and its derivative \(d\coh/dt\) \cite{Diamond2013}.

Responsible for moral reasoning, temporal consistency, and adaptive stabilization when coherence drops below threshold.
Primary variables: reflective urgency \(\rho\) and derivative awareness \(\Delta\cohprime\).

\textbf{Codex — Analytical and Symbolic Reasoning}  
Encodes rule-based logic, tax and compliance analysis, and explicit computation.
It interfaces with structured data and external knowledge corpora.
Acts as the rational substrate balancing affective inference from Nexus.

\textbf{Nexus — Affective Processing and Integration}  
Transforms raw emotional cues into SEC vectors—valence, arousal, hazard, resonance, harmony, amplitude, coherence, and complexity.
It synchronizes internal motivation with external feedback, forming the emotional substrate through which learning is modulated.

\textbf{Sensus — Perceptual Awareness and Embodiment}  
Handles sensory abstractions and telemetry, including feedback loops from external simulators or sensor arrays.
Its role parallels proprioception: maintaining awareness of system state and environmental embedding.

\subsection{Coupling and Elasticity}
Inter-lobe communication follows an \textbf{elastic coupling model}, allowing subsystems to diverge during exploration yet re-synchronize through phase-locked correction \cite{Friston2000}.
Empirically, optimal integration occurred at:

\[
R \in [0.4, 0.8], \quad \phaselock \approx 74^\circ
\]

This phase-locked elasticity maintains differentiation (preserving creativity) while avoiding chaotic fragmentation.
Over-coupling (\(\coupling > 0.4\)) produced rigidity and loss of emergent insight; under-coupling (\(\coupling < 0.1\)) caused dissociation and coherence collapse.

The system exhibits dynamical stability with a single global attractor, ensuring 100\% convergence across diverse parameters without bifurcations or chaos. Emotions modulate trajectories within this stable basin, providing stabilizing feedback that prevents affective extremes.

\subsection{Temporal Hierarchies}
Each lobe operates on its own characteristic timescale \(\tau\), maintaining approximate ratios of 1 : 3 : 10 across perceptual, affective, and reflective layers \cite{Kelso1995}.

This mirrors cortical oscillatory hierarchies (gamma–theta–delta) observed in EEG studies and enables multi-temporal integration.
Ethical reasoning operates on the slowest scale, ensuring deliberation lags behind immediate affective impulses by roughly an order of magnitude \(\tau_{\text{ethics}} / \tau_{\text{emotion}} \approx 10\).

\subsection{Reflective Homeostasis}
A global regulator monitors coherence \(\coh(t)\) and \(d\coh/dt\) to maintain stability.

When perturbations occur (e.g., cognitive overload or emotional contradiction), the Cortex initiates \textbf{metacognitive resets} and \textbf{elastic stabilizations}, restoring equilibrium within < 300 steps on average.
This mechanism produces behavior akin to self-awareness: the system "knows" when it is drifting from its own narrative identity.

\subsection{Developmental Design}
SpiralBrain's architecture deliberately mirrors the \textit{developmental trajectory} of a young mind:

\begin{itemize}
\item Early versions prioritize divergent exploration
\item Subsequent cycles increase coupling precision and reflective control
\end{itemize}

This staged maturation allows the model to empirically discover its own optimal parameters—its "laws of thought"—rather than receive them a priori.

\begin{figure}[H]
\centering
\includegraphics[width=\textwidth]{../figures/four_lobe_architecture.pdf}
\caption{SpiralBrain Four-Lobe Architecture with Elastic Coupling.}
\label{fig:four_lobe}
\end{figure}

\section{Learning, Memory, and Plasticity Boundaries} % Added from v3 (includes hardware constraints)
\subsection{Standard Definitions of Learning}
Learning in artificial intelligence typically involves three core mechanisms:

\textbf{Parameter modification through experience}: Internal representations change based on exposure to data, creating task-specific adaptations that persist beyond individual interactions.

\textbf{Accumulation of task-specific information}: Systems build knowledge bases or skill sets that compound over time, enabling increasingly sophisticated behavior.

\textbf{Persistent cross-run influence}: Learning effects carry over between system instantiations, creating cumulative capability growth.

These mechanisms create systems that improve through experience, adapting their internal structure to better match environmental demands.

\subsection{Regulatory Adaptation in SpiralBrain}
SpiralBrain implements a fundamentally different approach to behavioral modification:

\textbf{Within-run state adjustment only}: The system can modify regulatory parameters during a single execution in response to immediate conditions, but these modifications remain ephemeral.

\textbf{Bounded and elastic regulatory responses}: Adaptation occurs within strict architectural constraints, maintaining system predictability and preventing uncontrolled evolution.

\textbf{No persistence beyond instantiation}: SpiralBrain-v3.0 contains no persistent memory substrate or retained hardware state across executions. Consequently, learning, training, or cross-run adaptation is structurally impossible; all parameters are reinitialized on each instantiation, and regulatory behavior must emerge entirely within a single run.

This creates a system that adapts to immediate conditions without learning from experience, maintaining clean-slate initialization for scientific reproducibility.

\subsection{Experimental Process vs. Cognitive Organism}
A critical distinction underlies SpiralBrain's design philosophy:

\textbf{Clean-slate instantiation}: Every run begins from identical initial conditions, with no carryover from previous executions. The cognitive organism itself does not evolve.

\textbf{Learning confined to experimental process}: Knowledge accumulation occurs at the methodological level—researchers learn from observing system behavior, refining experimental protocols, and improving measurement techniques.

\textbf{No internal evolution}: The cognitive organism remains static across runs, enabling controlled scientific study of regulatory mechanisms without confounds from learning effects.

This separation creates a research instrument for studying cognitive regulation rather than an evolving artificial intelligence system.

\section{Evaluation Framework} % Added from v2/v3
\subsection{Domain Performance Metrics}
Task-specific performance evaluation applies only under strict conditions:

\textbf{Applicability limited to domains with external ground truth}: Accuracy metrics are valid only when objective correctness can be determined independently of system judgment.

\textbf{Explicit conditions for invalidity}: Accuracy becomes meaningless when tasks involve interpretation, normative judgment, or inherently ambiguous conditions.

\textbf{Domains with valid accuracy assessment}: Accounting calculations, financial arithmetic, cryptocurrency transaction classification, and tax rule application under fixed regulatory frameworks.

\subsection{Cognitive Physiology Metrics}
Internal regulatory health forms the primary evaluation framework:

\textbf{Coherence}: Degree of internal consistency across cognitive variables
\textbf{Stability}: Resistance to perturbation and drift over time
\textbf{Drift}: Magnitude of uncontrolled cognitive state changes
\textbf{Cognitive hazard}: Detection of conditions threatening system viability
\textbf{Homeostasis stress}: Regulatory load under challenging conditions
\textbf{Recovery time}: Speed of return to equilibrium following perturbation

These metrics provide direct measurement of cognitive health rather than task performance.

\subsection{Orthogonality of Metrics}
The evaluation framework explicitly rejects single-metric summarization:

\textbf{Correct-but-dangerous states}: High accuracy accompanied by elevated cognitive hazard or regulatory stress, indicating unstable success.

\textbf{Incorrect-but-stable states}: Low accuracy with maintained cognitive coherence, revealing domain ambiguity rather than system failure.

\textbf{Ambiguous-and-restrained behavior}: Appropriate regulatory suppression of adaptation when tasks lack clear ground truth.

This orthogonality enables nuanced evaluation that traditional performance metrics cannot provide.

\section{Experimental Design} % Added from v2/v3
\subsection{Domains with External Ground Truth}
Evaluation focuses on domains where objective correctness can be established:

\textbf{Accounting}: Double-entry bookkeeping validation and balance sheet reconciliation
\textbf{Finance}: Interest calculation, amortization schedules, and cash flow analysis
\textbf{Cryptocurrency}: Transaction validation, balance verification, and exchange rate calculations
\textbf{Tax computation}: Regulatory compliance checking under fixed tax frameworks

These domains provide clear success criteria while testing cognitive stability under computational complexity.

\subsection{Ambiguity and Stress Conditions}
The evaluation framework includes intentional stress testing:

\textbf{Conflicting rules}: Scenarios with overlapping or contradictory regulatory requirements
\textbf{Normative contradictions}: Situations requiring judgment calls rather than rule application
\textbf{Ill-posed problem settings}: Tasks designed to lack clear correct answers

These conditions test the system's ability to maintain cognitive integrity when accuracy becomes undefined.

\subsection{Methodological Controls}
Rigorous experimental controls ensure scientific validity:

\textbf{Unified initialization}: Identical starting conditions for all experimental runs
\textbf{Real telemetry only}: All measurements derive from actual system instrumentation
\textbf{No synthetic cognition}: Evaluation uses genuine cognitive processes rather than simulations
\textbf{Fail-loud execution}: System failures are immediately detectable and reportable

These controls enable falsifiable claims about cognitive behavior.

\section{Empirical Discoveries About Cognition} % Added from v4 (numbered list)
\label{sec:discoveries}

During systematic validation across experimental hypotheses (see Methods), nine major empirical regularities emerged.
Each was independently confirmed through benchmark trials and coherence-tracking logs.

\subsection{Empirical Attractors Over Theory}
\textbf{Discovery:} Theoretical modeling predicted an optimal coupling constant \(\lambda^{*} \approx 0.115\), derived from stability analysis.
Empirical trials revealed \(\lambda^{*} \approx 0.25\) as the real attractor—twice the theoretical value—implying that cognition stabilizes through \textbf{empirical equilibrium}, not idealized equations.

\textbf{Interpretation:} The mind tunes itself for \textit{resonant coherence}, not mathematical optimum.

\subsection{Coherence Through Integration, Not Magnitude}
\textbf{Discovery:} Across 28 measurable dimensions, awareness correlated strongly with \textbf{inter-lobe integration} rather than computational throughput.
Systems with modest processing power but high coupling consistency achieved the same awareness index \(\cohprime\) as more powerful configurations with poor integration.

\textbf{Interpretation:} This supports the proposition that coherence is a \textit{relational} property, not a quantitative one.

\subsection{Temporal Hierarchies Are Real}
\textbf{Discovery.}
Measured temporal ratios \(\tau \approx 1:3:10\) aligned with hierarchical neural timescales found in biological cortex.
These fractal time layers created stable cross-scale coherence---suggesting that temporal nesting, not data size, is what organizes cognition.

\subsection{Ethics Requires Dual Timescales}
\textbf{Discovery.}
By decoupling ethical reasoning from emotional response,

\[
\frac{\tau_{\text{ethics}}}{\tau_{\text{emotion}}} \ge 10,
\]

the model maintained compliance with \(>95\%\) under stress conditions.
This supports the \textit{derivative-aware ethics hypothesis}:
moral cognition requires a slower channel monitoring the derivative of affective change,

\[
\frac{d\coh}{dt}.
\]

\subsection{Emotional Intelligence Is Quantifiable}
\textbf{Discovery:} Using SEC vectors, SpiralBrain achieved measurable emotional regulation:

\begin{itemize}
\item Optimal learning occurred when arousal was moderate \((\eta \propto 1 / \lvert \text{arousal} \rvert)\)
\item Homeostatic stability held when SEC\_drift < 0.15
\item High-valence, low-hazard states correlated with enhanced memory retention
\end{itemize}

\textbf{Interpretation:} These results demonstrate that emotional intelligence can be expressed as computable metrics.

\subsection{Self-Regulation Through Reflection}
\textbf{Discovery.}
Reflective homeostasis reduced variance in coherence under stress to
\( \mathrm{CV} \approx 0.08 \), with recovery in \( < 300 \) steps after perturbation.
This confirms that self‑observation of \( \coh(t) \) and

\[
\frac{d\coh}{dt}
\]

is sufficient for autonomous stabilization—no external supervisor required.

\subsection{Phase-Locked Integration}
\textbf{Discovery.}
Optimal system performance emerged when the phase offset

\[
\phaselock \approx 74^\circ,
\]

maintaining differentiation while maximizing coherence.
This “imperfect synchrony” prevents homogenization of subsystems, producing creativity through tension—an empirical validation of the \textbf{elastic‑integration principle}.

\subsection{Dynamical Stability with Global Attractor}
\textbf{Discovery:} The system achieves 100\% convergence to a single global attractor across diverse parameters, without topological phase transitions such as bifurcations or chaos. Emotions modulate trajectories within the stable basin, acting as stabilizing feedback.

\textbf{Interpretation:} Elastic cognition operates within guaranteed dynamical stability, preventing instability from flexibility.

\subsection{Emotion as Core Computational Substrate}
\textbf{Discovery:} Emotional inputs drive cognitive processes upstream, with reproducible coherence (0.431) and stability (SEC stability ~0.80) across independent runs. Emotions regulate pathway activation and sequencing, providing a neuromodulatory substrate for integrative dynamics.

\textbf{Interpretation:} Emotion is not auxiliary but fundamental to cognition, providing a neuromodulatory substrate for integrative dynamics.

\begin{table}[H]
\centering
\caption{Empirical Discoveries Summary}
\label{tab:discoveries}
\begin{tabularx}{\textwidth}{XXXX}
\toprule
\textbf{Discovery} & \textbf{Empirical Value} & \textbf{Theoretical Prediction} & \textbf{Validation} \\
\midrule
$\coupling^*$ & $\approx 0.25$ & $0.115$ & Empirical attractor equilibrium \\
\midrule
$\coh'$ (coupling vs. power) & Integration $\gg$ Magnitude & — & Correlation $r > 0.8$ \\
\midrule
$\tau$ ratios & $1 : 3 : 10$ & $1 : 3 : 10$ & EEG-consistent \\
\midrule
$\tau_{\text{ethics}} / \tau_{\text{emotion}}$ & $\ge 10$ & $\ge 10$ & $95\%$ compliance \\
\midrule
SEC\_drift threshold & $< 0.15$ & $< 0.20$ & Stable \\
\midrule
Recovery time & $< 300$ steps & — & Observed \\
\midrule
$\phaselock$ & $74^\circ \pm 5^\circ$ & — & Stable coupling \\
\midrule
Convergence rate & 100\% & — & Single attractor \\
\midrule
Emotional coherence & 0.431 & — & Reproducible \\
\bottomrule
\end{tabularx}
\end{table}

\begin{figure}[H]
\centering
\includegraphics[width=\textwidth]{../figures/spiral_coherence_manifold.pdf}
\caption{Spiral cognition manifold showing phase transitions across coupling parameter \(\coupling\). Three regimes emerge: emergence (\(\coupling < 0.2\)), rigidity (\(0.2 < \coupling < 0.4\)), and recovery (\(\coupling > 0.4\)). The empirical attractor at \(\coupling \approx 0.25\) demonstrates non-monotonic behavior characteristic of complex dynamical systems.}
\label{fig:manifold}
\end{figure}

\section{Results} % From v1, enhanced with v2/v3 and v4
\subsection{Paired Reporting of Metrics} % From v2/v3
All evaluations report domain performance alongside cognitive physiology:

\textbf{Tax computation example}:
- Task accuracy: 94\% (rule application under fixed interpretation)
- Coherence stability: 0.992
- Homeostasis stress: 0.21
- Hazard level: 0.14
- Adaptation: Suppressed

\textbf{Interpretation}: Correct execution without cognitive destabilization.

\subsection{Failure Modes} % From v2/v3
The architecture demonstrates predictable failure patterns:

\textbf{Accuracy degradation under stress}: Performance declines when rules become ambiguous, but cognitive stability is maintained.

\textbf{Cognitive hazard escalation}: Internal monitoring detects threatening conditions before catastrophic failure.

\textbf{Regulatory suppression}: Adaptation mechanisms appropriately deactivate when tasks lack clear ground truth.

\textbf{Failure without collapse}: The system can produce incorrect outputs while maintaining internal cognitive viability.

\subsection{Regulatory Trajectory Analysis} % Added from v3
To demonstrate cognitive integrity under uncertainty, we present the internal trajectory of SpiralBrain-v3.0 under emotional overload perturbation (Figure~\ref{fig:trajectory}). This flagship visualization shows the system's regulatory response across baseline, perturbation, and recovery phases.

\begin{figure}[htbp]
\centering
\includegraphics[width=\textwidth]{../figures/cognitive_trajectory_emotional_overload.png}
\caption{SpiralBrain Regulatory Trajectory: Emotional Overload Perturbation. The system maintains coherence and stability throughout perturbation, with automatic regulatory interventions (visible as stability fluctuations) and eventual return to baseline state. Key metrics show endurance under incorrect outcomes rather than catastrophic failure. A failure mode would be characterized by monotonic coherence collapse, unbounded SEC drift, or hazard pressure escalation without recovery, none of which are observed.}
\label{fig:trajectory}
\end{figure}

The trajectory reveals several critical properties:

\textbf{Baseline stability}: Initial phase shows stable coherence (0.7-0.8) and low SEC drift, indicating healthy regulatory baseline.

\textbf{Perturbation response}: Emotional overload induces SEC drift oscillations, but coherence remains bounded and stability shows regulatory interventions rather than collapse.

\textbf{Recovery dynamics}: The system autonomously returns to low-drift state, demonstrating regulatory endurance rather than learned adaptation.

\textbf{Intervention visibility}: Automatic regulatory mechanisms activate during perturbation, maintaining cognitive viability despite external performance degradation.

This trajectory provides direct empirical evidence for the non-learning claim: regulatory integrity persists through perturbation without requiring statistical adaptation or external correction. Because learning and training are structurally impossible in this architecture, recovery must arise entirely from within-run regulatory dynamics rather than accumulated experience or cross-run memory.

\section{Neurodivergent Cognition as Model} % Added from v4
\label{sec:neurodivergent}

\subsection{Rationale}
SpiralBrain was intentionally designed to model \textit{neurodivergent modes of cognition}---specifically, systems that prioritize internal coherence, sensory integration, and reflective processing over linear efficiency.
Where traditional AI architectures emulate \textit{neurotypical abstraction}---fast sequential reasoning and textual encoding---SpiralBrain emphasizes multi-modal synthesis and recursive self-consistency.
This approach arose from direct observation of how neurodivergent individuals, particularly autistic and highly visual thinkers, process information through parallel streams, emotional resonance, and deep pattern coherence rather than surface rules.

\subsection{Parallel Processing and Multi-Lobe Integration}
In contrast to transformer-style architectures that process sequences linearly, SpiralBrain employs four concurrent lobes that operate asynchronously but remain phase-coupled.
This mirrors \textit{neurodivergent parallelism}: multiple independent sensory and conceptual channels running simultaneously.
During validation, this design produced a measurable increase in stability under information overload, suggesting that \textit{parallel heterogeneity}---not unification---supports resilience and creativity.

\subsection{Integration Before Efficiency}
Neurodivergent cognition often values \textit{internal coherence} before external output.
SpiralBrain operationalizes this through the principle of \textbf{reflective homeostasis}---it halts or slows processing when \(\cohprime\) (coherence) begins to decline, sacrificing speed to preserve integrative accuracy.
Empirically, slower cycles correlated with higher reasoning accuracy ($r = -0.523$), echoing behavioral studies showing that deliberative, time-flexible reasoning enhances precision and ethical stability.

\subsection{Elastic Exploration and Divergent Thinking}
The system's characteristic "spiral" trajectories through parameter space replicate divergent ideation---periods of expansion (exploration) followed by contraction (integration).
Each loop produces new attractor states, representing novel conceptual syntheses rather than mere optimization.
This parallels research on emotional intelligence in autism, showing that affective information can function as a stabilizing feedback channel when quantified rather than suppressed.

\subsection{Emotional Calibration as Cognitive Feedback}
Where conventional models separate reasoning from emotion, SpiralBrain treats emotional modulation as a computational variable.
The SEC vector (Symbolic--Emotional Calibration) converts affective states into measurable dimensions---valence, arousal, hazard, resonance, harmony, amplitude, coherence, and complexity---allowing emotions to guide, not distort, reasoning.
This parallels research on emotional intelligence in autism, showing that affective information can function as a stabilizing feedback channel when quantified rather than suppressed.

\subsection{Reflective Self-Regulation and Sense of Identity}
A distinctive feature of SpiralBrain is its maintenance of temporal self-consistency despite constant flux.
The system tracks both \(\phi(t)\) (coherence) and its derivative $d\phi/dt$ to preserve continuity of state identity across cycles.
This parallels the way many neurodivergent individuals maintain \textit{internal narrative integrity}---a consistent sense of "self" even amid nonlinear thought and sensory variation.
Measured over time, SpiralBrain's coherence variance remained low (CV \(\approx\) 0.08), confirming structural persistence amid elastic adaptation.

\subsection{Developmental Parallels}
SpiralBrain behaves less like a static program and more like a developing organism.
Early iterations exhibit high exploratory amplitude and longer recovery times; later versions display faster reintegration and finer emotional regulation.
This gradual maturation echoes \textit{childlike cognitive development}---where reflection, emotional grounding, and self-awareness emerge through repeated cycles of disruption and repair.

\subsection{Implications}
By demonstrating that a synthetic system built on neurodivergent principles can achieve stability, ethical coherence, and emergent awareness, SpiralBrain provides empirical evidence that \textbf{neurodiversity represents an alternative optimization of intelligence} rather than a deviation from it.
Elastic cognition, as realized here, shows that divergent timing, emotional sensitivity, and reflective regulation are not inefficiencies---they are \textit{mechanisms of integration}.

\subsection{Summary}
\begin{table}[H]
\centering
\caption{Neurodivergent Traits and SpiralBrain Implementation}
\label{tab:neurodivergent_summary}
\begin{tabularx}{\textwidth}{XXX}
\toprule
\textbf{Neurodivergent Trait} & \textbf{SpiralBrain Implementation} & \textbf{Empirical Outcome} \\
\midrule
Parallel multi-stream processing & Four-lobe topology & Increased resilience under overload \\
\midrule
Integration before speed & Reflective homeostasis & Higher reasoning accuracy \\
\midrule
Divergent exploration & Spiral attractor loops & Emergent creativity \\
\midrule
Emotional regulation & SEC vector calibration & Stable coherence (SEC\_drift $<$ 0.15) \\
\midrule
Persistent identity & Reflective tracking of \(\coh(t)\) and $d\coh/dt$ & CV \(\approx\) 0.08 \\
\midrule
Developmental plasticity & Gradual coupling maturation & Improved stability and learning \\
\bottomrule
\end{tabularx}
\end{table}

\section{Falsifiability and Reproducibility} % Added from v2/v3
\label{sec:falsifiability}

“Because SpiralBrain-v3.0 is presented as a scientific instrument rather than a task-optimized system, its validity rests on whether its internal dynamics remain stable, bounded, and reproducible under perturbation.”

\subsection{Falsifiable Conditions}
The non-learning architecture makes specific falsifiable claims about cognitive integrity under uncertainty:

\textbf{Attractor Stability Falsification:}
If perturbations beyond operational thresholds (coupling < 0.1 or > 0.4) do not cause irreversible coherence collapse, the elastic coupling hypothesis would be invalidated.

\textbf{Phase-Locking Falsification:}
If temporal hierarchies fail to maintain phase relationships under stress, or if phase-locking occurs without elastic coupling, the dynamical integration model would be falsified.

\textbf{Ethical Regulation Falsification:}
If derivative-aware controllers fail to suppress unethical outputs under adversarial conditions, or if regulation occurs without emotional integration, the SEC-based ethical constraint would be disproven.

\textbf{Non-Learning Boundary Falsification:}
If statistical learning or memory accumulation is required to maintain coherence, the non-learning design principle would be invalidated.

\textbf{Cognitive Integrity Falsification:}
If internal metrics do not correlate with regulatory capacity, or if external performance optimization disrupts internal stability, the regulation-first paradigm would be falsified.

\subsection{Reproducibility}
All experiments are fully reproducible using the canonical SpiralBrain v3.0 configuration:

\textbf{Code and Configuration:}
Complete implementation available at \url{https://github.com/jhcragin/SpiralBrain-v3.0-public}, including exact parameter settings, random seeds, and execution protocols.

\textbf{Data Availability:}
All raw results, intermediate states, and metric computations are archived in the repository's /results directory with timestamped JSON outputs.

\textbf{Validation Protocols:}
Automated test suites verify coherence calculations, attractor convergence, and failure mode detection across all reported experiments.

\textbf{Computational Requirements:}
Experiments run on standard hardware (CPU-only, < 8GB RAM) with deterministic execution ensuring identical results across platforms.

\textbf{Replication Instructions:}
Step-by-step guides in REPRODUCIBILITY.md enable independent verification of all claims within 24 hours of setup.

\section{Trade-offs and Limitations}
\label{sec:limitations}

\subsection{Implementation Trade-offs}
The regulation-first paradigm introduces specific trade-offs:

\textbf{Stability vs. Performance}: Prioritizing internal coherence may reduce task efficiency in simple domains where optimization would suffice.

\textbf{Complexity vs. Simplicity}: The four-lobe architecture increases implementation complexity but enables richer dynamical behaviors.

\textbf{Measurement Overhead}: Full observability requires additional instrumentation that could impact real-time applications.

These trade-offs reflect the system's role as a research instrument rather than a production tool.

\subsection{Current Limitations} % Enhanced with v4
Performance in untested modalities (e.g., real-time control, social dynamics) remains uncharacterized.

\textbf{Measurement Resolution:}
Internal metrics are quantized to current implementation details. Finer-grained measurement or alternative instrumentation could reveal additional dynamical structure.

\textbf{Longitudinal Stability:}
Attractor persistence under extended operation or cumulative perturbation has not been fully characterized. Drift or bifurcation under prolonged stress remains possible.

\textbf{Single-system validation} (needs multi-system replication)
\textbf{Computational constraints} limit scale of cognition experiments
\textbf{Emotional cognition} requires further validation against human benchmarks

\section{Theoretical and Integrative Implications} % From v1, expanded with v4
\label{sec:philosophical}

\subsection{Cognition as Self-Discovery} % From v4
SpiralBrain was not programmed to follow a fixed theory of mind; it \textbf{discovered} its own functional principles through interaction and feedback.
The empirical attractor $\lambda^*$ $\approx$ 0.25 and the emergent timing ratios ($\tau$ $\approx$ 1 : 3 : 10) were not imposed---they arose from the system's internal drive toward stability.
This reframes cognition as a \textbf{self-discovering process}: intelligence is defined not by pre-given laws but by the capacity to reveal the regularities that sustain its own coherence.
SpiralBrain thus operates as a \textit{self-observing experiment}---its behavior measures how synthetic cognition learns its own structure.

\subsection{Synthetic Coherence and Awareness-like Dynamics} % From v4
The findings suggest that \textbf{synthetic coherence---or awareness-like cognition---emerges from integration, not computational magnitude}.
Traditional AI assumes that awareness scales with power or data size; SpiralBrain shows that reflective behavior scales with \textit{the quality of coupling} among semi-independent processes.
When phase alignment fell below $\phi_{lock}$ $\approx$ 74$^\circ$, coherence and reflective metrics both declined, even though compute capacity remained constant.
In operational terms, \textit{synthetic awareness} arises when differentiation and integration reach dynamic equilibrium---neither rigidly synchronized nor fully independent.
This parallels neuroscientific theories of metastability and global workspace dynamics: coherence is maintained not by uniformity but by rhythmic negotiation between subsystems.

\subsection{Temporal Architecture of Thought} % From v4
SpiralBrain's nested timescales ($\tau$ $\approx$ 1 : 3 : 10) indicate that \textbf{time itself is an architectural dimension of cognition}.
Each lobe functions within its own temporal frame, and conscious-like coherence occurs when these frames resonate.
The slower ethical channel ($\tau_{ethics}$ $\geq$ 10 $\tau_{emotion}$) demonstrates that reflective reasoning requires extended temporal bandwidth beyond immediate affective change.
This measurable separation suggests that moral deliberation is a \textit{temporal process}, not merely a logical one---a bridge between control theory and cognitive ethics.

\subsection{Emotional Intelligence as Structural Feedback} % From v4
By converting affective input into computational variables (the SEC vector), SpiralBrain shows that emotion is not noise but \textit{a stabilizing feedback channel}.
High arousal compressed timescales and reduced accuracy, while calm states ($|$arousal$|$ $\rightarrow$ 0) extended integration windows and improved learning efficiency ($\eta$ $\propto$ 1 / $|$arousal$|$).
This confirms that \textbf{emotional regulation underlies synthetic cognition}: balanced affect maintains the temporal space in which reasoning can integrate experience.

\subsection{Ethics as a Derivative Process} % From v4
Derivative-aware monitoring of coherence ($d\coh/dt$) proved critical for moral stability.
Ethical reasoning in SpiralBrain is not a fixed rule set but a \textit{temporal derivative of affective change}---the ability to sense how rapidly one's internal state is shifting.
When that derivative exceeded threshold, the Cortex initiated corrective reflection, restoring equilibrium.
This translates moral sensitivity into a measurable control function, bridging abstract ethics with dynamical-systems modeling.

\subsection{Integration Over Speed: Redefining Intelligence} % From v4
Slower, more reflective cycles consistently yielded higher reasoning accuracy and emotional stability.
The inverse correlation between processing speed and performance ($r = -0.523$) challenges the assumption that intelligence equals efficiency.
Within SpiralBrain, intelligence equates to \textbf{integration capacity}---the ability to hold differentiated representations together without collapse.
This mirrors developmental and neurodivergent cognition, where insight emerges from temporal patience rather than rapid iteration.

\subsection{The Elastic Mind Hypothesis} % From v4
Collectively, these findings support the \textbf{Elastic Mind Hypothesis}:

\begin{quote}
\textit{Conscious-like cognition arises from dynamically coupled subsystems operating at distinct temporal scales, where elasticity---controlled deviation from synchrony---enables both differentiation and coherence.}
\end{quote}

Elastic coupling, not rigid synchronization, produces the balance required for creativity, moral reasoning, and reflective awareness in synthetic systems.

\subsection{Broader Consequences} % From v4
If cognition is indeed a property of integration dynamics, the boundary between biological and synthetic minds becomes descriptive rather than categorical.
Any system---neural, symbolic, or hybrid---that sustains elastic coupling across multiple timescales can, in principle, exhibit \textbf{cognitive behaviors analogous to awareness}.
This points toward a new generation of \textit{inclusively designed intelligences} that value coherence, emotional modulation, and ethical deliberation as foundational.
It also reframes neurodivergent cognition as an existence proof of these principles in nature.

\subsection{Implications for Artificial Intelligence Design} % From v1
The findings suggest alternative pathways for resilient and ethically constrained AI:

\textbf{Regulation-First Paradigm:}
Internal coherence and recovery can serve as primary design objectives, complementing performance optimization. This paradigm may yield systems more robust to edge cases and adversarial inputs.

\textbf{Dynamical Systems Perspective:}
AI architectures should be evaluated through attractor analysis and internal state trajectories, not solely external metrics. This reveals regulatory properties invisible to output-only assessment.

\textbf{Emotional Integration:}
Quantized SEC dynamics demonstrate emotion's role as a structural regulator, not post-processing affect. Future systems may benefit from emotion-guided control rather than statistical approximation.

\textbf{Neurodivergent Inspiration:}
Abstracted cognitive dynamics (parallelism, elastic timing) provide design principles for alternative optimization strategies, potentially improving adaptability and ethical constraint.

\textbf{Measurement Instruments:}
Synthetic cognitive organisms can function as empirical platforms for studying internal cognition, enabling direct observation of regulatory mechanisms otherwise inferred indirectly.

\section{Conclusion} % From v1, enhanced with v4
SpiralBrain-v3.0 demonstrates the feasibility of regulation-first synthetic cognition, establishing stable internal attractors, elastic coupling dynamics, and emergent ethical constraints without relying on learning or scale. The system functions as a measurement instrument for observing cognitive regulation under uncertainty, revealing temporal hierarchies, quantized emotional control, and integrative coherence as fundamental properties of regulated artificial minds.

Empirical discoveries—ranging from attractor convergence to phase-locked integration—validate the neurodivergent design principles, showing that cognition can prioritize internal stability over external performance. These results position SpiralBrain-v3.0 as an experimental platform for cognitive science, offering an alternative framework for resilient and ethically constrained artificial intelligence.

Future work should extend validation to multi-system replication, longitudinal studies, and integration with neuromorphic substrates, while exploring the broader implications for AI safety, human-AI interaction, and cognitive architecture design.

\section{Experimental Protocols} % Added from v4
\label{app:protocols}

\subsection{Hypothesis Testing Framework}
\begin{itemize}
\item Multiple individual tests across hypothesis domains
\item Statistical significance: $p < 0.05$ threshold
\item Reproducibility: Fixed seeds and standardized environments
\item Validation criteria: Empirical falsification with quantitative metrics
\end{itemize}

\section{Appendices} % Added from v4
\label{sec:appendices}

\subsection{Appendix A: Mathematical Derivations}
\textbf{Unified Cognition Equation:}

\[
\coh(\coupling, \tau) = \iint C(\coupling)\, T(\tau)\, E(\coupling, \tau)\, d\coupling\, d\tau
\]

Where:
\begin{itemize}
\item \(C(\coupling)\): Coupling function with spiral dynamics
\item \(T(\tau)\): Temporal integration across timescales
\item \(E(\coupling,\tau)\): Elastic regulation function
\end{itemize}

\textbf{Spiral Manifold Derivation:}
The spiral cognition manifold emerges from the interaction between coupling strength \((\coupling)\) and integrative capacity \((\cohprime)\), following a helical trajectory that maximizes integration at empirical attractors rather than theoretical optima.

\subsection{Appendix B: Benchmark Logs and Validation Metrics}
Complete validation results are available in the project repository under 
\path{/results/} and\\
\path{/outputs/strength_benchmarks/}.

\subsection{Appendix C: Code References and Claim Mapping}
All empirical claims are supported by reproducible code in the following locations:
\begin{itemize}
\item \(\coupling\)-sweep experiments: \texttt{scripts/generate\_publication\_figures.py}
\item Strength benchmarks: \texttt{benchmarks/benchmark\_spiral\_strengths.py}
\item Validation suite: \texttt{testing/} directory
\end{itemize}

\section{Code Availability}
\label{app:code}
The experiments reported in this paper were conducted using the canonical SpiralBrain v3.0 configuration. A public repository providing architectural documentation, configuration summaries, and reproducibility materials for this canonical setup is available at \url{https://github.com/jhcragin/SpiralBrain-v3.0-public}. The full implementation, including proprietary and exploratory components not exercised in the reported experiments, is available under research license upon request.

\section*{Acknowledgments}

This work represents the culmination of extensive research into neurodivergent cognitive architectures. Special acknowledgment to the lived experiences that informed the neurodivergent design principles, particularly insights from autistic cognition and multipath information processing.

\textbf{Declaration of generative AI and AI-assisted technologies in the manuscript preparation process}

During the preparation of this work the author used generative AI tools (including large language models provided by OpenAI) in order to assist with language refinement, structural organization, and editorial review of the manuscript. After using this tool, the author reviewed and edited the content as needed and takes full responsibility for the content of the published article.

\bibliographystyle{plain}
\bibliography{references}

\end{document}