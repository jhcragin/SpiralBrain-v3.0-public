\documentclass[11pt]{article}
\usepackage[margin=1in]{geometry}
\usepackage{graphicx}
\usepackage{hyperref}

\begin{document}

\title{Emotion as a Control Signal for Symbolic Stability in Neurosymbolic Systems}
\author{John H. Cragin}
\date{\today}

\maketitle

\begin{abstract}
This paper investigates emotional regulation as a control signal that enhances symbolic stability in neurosymbolic systems. Using SpiralBrain v3.0 as an instrumented testbed, we test the hypothesis that symbolic coherence generally declines with increasing SEC drift under stress conditions, and recovery proceeds more rapidly with affective stabilization than through autonomous symbolic repair alone. Falsification requires demonstrating no correlation between SEC drift and coherence degradation, or equivalent recovery rates with and without emotional intervention.
\end{abstract}

\section{Introduction}

Neurosymbolic systems combine the pattern recognition capabilities of neural networks with the interpretability and logical reasoning of symbolic processing. However, maintaining symbolic stability under varying conditions remains a significant challenge. Traditional approaches to stability focus on algorithmic improvements or architectural modifications, but these often overlook the role of affective states in modulating cognitive processes.

This paper explores the hypothesis that emotional regulation serves as a critical control signal for maintaining symbolic stability in neurosymbolic architectures. By examining SpiralBrain v3.0, a system with explicit emotional processing components, we investigate how affective states influence symbolic coherence and recovery from stress-induced instability.

\section{Problem Statement}

Symbolic processing in neurosymbolic systems can exhibit instability when subjected to environmental stress or internal perturbations. This manifests as degraded coherence between symbolic representations, leading to inconsistent reasoning and unreliable outputs. Traditional stabilization methods, such as increased computational resources or algorithmic refinements, may not address the underlying affective factors that contribute to this instability.

The problem is particularly acute in systems that must maintain reliable symbolic processing across diverse emotional contexts. Without proper affective regulation, symbolic stability becomes contingent on external conditions rather than being an intrinsic property of the system.

\section{Hypothesis and Falsification}

\textbf{Hypothesis:} Emotional regulation enhances symbolic stability in neurosymbolic systems. Specifically:
1. Symbolic coherence generally declines with increasing SEC drift under stress conditions.
2. Recovery from instability proceeds more rapidly with affective stabilization than through autonomous symbolic repair alone.

\textbf{Falsification Criteria:}
The hypothesis is falsified if any of the following are demonstrated:
1. No correlation between SEC drift and symbolic coherence degradation.
2. Equivalent recovery rates and final coherence levels with and without regulatory intervention.
3. Consistent symbolic stability across all tested scenarios regardless of SEC drift.

\section{Methods}

Experiments were conducted using SpiralBrain v3.0, a neurosymbolic architecture with explicit emotional processing components. The system was subjected to controlled stress conditions designed to induce symbolic instability. Two experimental conditions were compared:

1. \textbf{Regulatory Intervention:} Active emotional regulation through SEC feedback loops.
2. \textbf{No Intervention:} Passive observation without affective stabilization.

Symbolic coherence was measured as the normalized correlation between pathway activations, ranging from 0.0 (complete incoherence) to 1.0 (perfect coherence). SEC drift was quantified as the Euclidean distance from baseline emotional state vectors.

Data was collected across multiple trials with varying stress intensities. Statistical analysis included correlation coefficients and trajectory comparisons between intervention and non-intervention conditions.

\section{Results}

\subsection{Symbolic Coherence as a Function of SEC Drift}

\begin{figure}[h]
\centering
\includegraphics[width=0.8\textwidth]{../figures/fig1_coherence_vs_sec_drift.png}
\caption{Symbolic coherence as a function of SEC drift. Data aggregated from emotional intervention logs.}
\label{fig:coherence_drift}
\end{figure}

Figure 1 plots symbolic coherence as a function of SEC drift using data aggregated from emotional intervention logs. SEC drift values span the range 0.0–0.5, while symbolic coherence values span 0.3–0.8.

The data shows a general trend of declining coherence with increasing SEC drift, though with considerable variability. No clear inflection point or threshold is observed in the relationship. For instance, relatively high coherence values (around 0.85-0.87) appear at both low (0.03) and moderate (0.28) drift levels, indicating that the relationship is not strictly monotonic.

Data points associated with regulatory intervention show SEC drift changes from 0.0 to 0.3, with corresponding coherence values changing from 0.7 to 0.5. The observed patterns suggest that while SEC drift correlates with coherence degradation on average, individual cases show substantial variation that may depend on specific intervention contexts.

\subsection{Coherence Recovery With and Without Regulatory Intervention}

\begin{figure}[h]
\centering
\includegraphics[width=0.8\textwidth]{../figures/fig2_recovery_trajectories.png}
\caption{Coherence recovery trajectories with and without regulatory intervention.}
\label{fig:recovery}
\end{figure}

Figure 2 compares coherence trajectories following stress onset for trials with regulatory intervention and trials without intervention. Time is indexed relative to stress onset at t=0, and coherence values are reported over 0–9 time steps.

Both regulated and unregulated trials start at coherence 0.2 and recover to 1.0 by the end of the observation period. However, the trajectories differ in their recovery dynamics: regulated trials show faster convergence, with the difference becoming visible from time step 2 onward. These trajectories are illustrative of the recovery patterns observed in the experimental data, suggesting that emotional regulation enhances the speed of symbolic recovery rather than enabling it entirely.

\section{Discussion}

The results provide evidence that emotional regulation enhances the performance of symbolic stability in neurosymbolic systems, though the data do not support the stronger claim that it acts as an absolute gating mechanism. The observed general decline in coherence with increasing SEC drift, despite variability, suggests that affective states influence symbolic processing stability. However, the lack of clear thresholds and the presence of high coherence at moderate drift levels indicate that the relationship is more nuanced than a simple gating function.

The recovery trajectories show that both regulated and unregulated systems can achieve full coherence (1.0), but emotional regulation accelerates this process. This suggests that autonomous symbolic repair mechanisms are functional, but affective stabilization provides a performance advantage by speeding recovery. This finding positions emotional regulation as an optimizer of symbolic processing rather than a prerequisite for basic stability.

These results have implications for the design of neurosymbolic systems, suggesting that incorporating emotional components can improve robustness and recovery speed under stress. The findings support treating affective states as important modulators of cognitive performance, potentially leading to more resilient architectures.

Limitations include the specific context of SpiralBrain v3.0 and the controlled experimental conditions. The observed variability in coherence-drift relationships suggests that contextual factors may play significant roles. Further research is needed to understand the generalizability of these findings and to identify the specific mechanisms through which emotional regulation influences symbolic processing.

\section{Conclusion}

This investigation demonstrates that emotional regulation enhances the performance of symbolic stability in neurosymbolic systems. Using SpiralBrain v3.0 as a testbed, we observed that symbolic coherence generally declines with increasing SEC drift, though with considerable variability and no clear thresholds. Recovery to full coherence occurs in both regulated and unregulated conditions, but emotional intervention accelerates this process.

The findings support the view that emotion acts as a performance enhancer for symbolic processing rather than an absolute gating mechanism. While autonomous symbolic repair is functional, affective stabilization provides advantages in speed and consistency of recovery. This has important implications for designing more robust neurosymbolic systems that leverage emotional components to optimize cognitive performance under stress.

Future work should explore the specific mechanisms of emotional-symbolic interaction, test generalizability across different architectures, and investigate how emotional regulation might be optimized for various application domains.

\bibliographystyle{plain}
\bibliography{references}

\end{document}