\documentclass[11pt]{article}
\usepackage[margin=1in]{geometry}
\usepackage{graphicx}
\usepackage{hyperref}

\begin{document}

\title{Emotion as a Control Signal for Symbolic Stability in Neurosymbolic Systems}
\author{John H. Cragin}
\date{\today}

\maketitle

\begin{abstract}
This paper investigates emotional regulation as a control signal that gates symbolic stability in neurosymbolic systems. Using SpiralBrain v3.0 as an instrumented testbed, we test the hypothesis that symbolic coherence degrades rapidly under stress when SEC drift exceeds empirically determined thresholds, and recovery proceeds through affective stabilization rather than autonomous symbolic repair. Falsification requires demonstrating consistent symbolic stability without regulatory intervention across tested scenarios.
\end{abstract}

\section{Introduction}

\section{Problem Statement}

\section{Hypothesis and Falsification}

\section{Methods}

\section{Results}

\subsection{Symbolic Coherence as a Function of SEC Drift}

Figure 1 plots symbolic coherence as a function of SEC drift using data aggregated from emotional intervention logs. SEC drift values span the range 0.0–0.5, while symbolic coherence values span 0.3–0.8.

Across the observed range, symbolic coherence exhibits decline as SEC drift increases. No inflection or threshold is observed in the data.

Data points associated with regulatory intervention show SEC drift changes from 0.0 to 0.3, with corresponding coherence values changing from 0.7 to 0.5. No additional relationships are reported beyond those visible in Figure 1.

\subsection{Coherence Recovery With and Without Regulatory Intervention}

Figure 2 compares coherence trajectories following stress onset for trials with regulatory intervention and trials without intervention. Time is indexed relative to stress onset at t=0, and coherence values are reported over 0–9 time steps.

In trials with regulatory intervention, coherence values change from 0.2 to 1.0 over the observed interval. In trials without regulatory intervention, coherence values change from 0.2 to 1.0.

The difference between regulated and unregulated trajectories is visible from time step 2 onward, with regulated trials exhibiting faster convergence. No claims beyond the plotted trajectories are made.

\section{Discussion}

\section{Conclusion}

\bibliographystyle{plain}
\bibliography{references}

\end{document}